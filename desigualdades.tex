%%%%%%%%%%%%%%%%%%%%%%%%%%%%%%%%%%%%%%%%
% Short Sectioned Assignment
% LaTeX Template
% Version 1.0 (5/5/12)
%
% This template has been downloaded from:
% http://www.LaTeXTemplates.com
% and has been modified. 
%
% Original author:
% Frits Wenneker (http://www.howtotex.com)
%
% License:
% CC BY-NC-SA 3.0 (http://creativecommons.org/licenses/by-nc-sa/3.0/)
%
%%%%%%%%%%%%%%%%%%%%%%%%%%%%%%%%%%%%%%%%%

%----------------------------------------------------------------------------------------
%	PACKAGES AND OTHER DOCUMENT CONFIGURATIONS
%----------------------------------------------------------------------------------------

\documentclass[paper=a4, fontsize=11pt, spanish]{scrartcl} % A4 paper and 11pt font size

\usepackage[T1]{fontenc} % Use 8-bit encoding that has 256 glyphs
\usepackage{fourier} % Use the Adobe Utopia font for the document - comment this line to return to the LaTeX default
\usepackage[spanish]{babel} % English language/hyphenation
\selectlanguage{spanish}
\usepackage[utf8]{inputenc}
\usepackage{amsmath,amsfonts,amsthm} % Math packages

\usepackage{sectsty} % Allows customizing section commands
\allsectionsfont{\centering \normalfont\scshape} % Make all sections centered, the default font and small caps

\usepackage{fancyhdr} % Custom headers and footers
\pagestyle{fancyplain} % Makes all pages in the document conform to the custom headers and footers
\fancyhead{} % No page header - if you want one, create it in the same way as the footers below
\fancyfoot[L]{} % Empty left footer
\fancyfoot[C]{} % Empty center footer
\fancyfoot[R]{\thepage} % Page numbering for right footer
\renewcommand{\headrulewidth}{0pt} % Remove header underlines
\renewcommand{\footrulewidth}{0pt} % Remove footer underlines
\setlength{\headheight}{13.6pt} % Customize the height of the header

\numberwithin{equation}{section} % Number equations within sections (i.e. 1.1, 1.2, 2.1, 2.2 instead of 1, 2, 3, 4)
\numberwithin{figure}{section} % Number figures within sections (i.e. 1.1, 1.2, 2.1, 2.2 instead of 1, 2, 3, 4)
\numberwithin{table}{section} % Number tables within sections (i.e. 1.1, 1.2, 2.1, 2.2 instead of 1, 2, 3, 4)

\setlength\parindent{0pt} % Removes all indentation from paragraphs - comment this line for an assignment with lots of text

%----------------------------------------------------------------------------------------
%	TITLE SECTION
%----------------------------------------------------------------------------------------

\newcommand{\horrule}[1]{\rule{\linewidth}{#1}} % Create horizontal rule command with 1 argument of height

\title{	
\normalfont \normalsize 
\textsc{Universidad de Granada.} \\ [25pt] % Your university, school and/or department name(s)
\horrule{0.5pt} \\[0.4cm] % Thin top horizontal rule
\huge Desigualdades \\ % The assignment title
\horrule{2pt} \\[0.5cm] % Thick bottom horizontal rule
}

\author{Doble Grado en Ingeniería Informática y Matemáticas} % Your name

\date{\normalsize\today} % Today's date or a custom date

\begin{document}

\maketitle % Print the title

%----------------------------------------------------------------------------------------
% Escribir una ecuación:
%  \begin{align} 
%  \begin{split}
%----------------------------------------------------------------------------------------
\newcommand {\equality}[1]{\textbf{Caso de igualdad:} {#1}}


\section{Desigualdades en espacios vectoriales.}
 \subsection{Desigualdad de Cauchy-Schwarz.}
  Para todo par de vectores $x,y$ en un espacio prehilbertiano:
  \begin{align}
   |{\langle x,y \rangle}| \leq \|x\|\|y\|
  \end{align}
  \equality{$x,y$ son linealmente dependientes.}
  
  \subsection{Desigualdad triangular.}
  Para todo par de vectores $x,y$ de un espacio vectorial normado:
  \begin{align}
    \displaystyle \|x+y\| \leq \|x\|+\|y\|
  \end{align}
  \equality{$x,y$ son de la forma $y = \lambda^2x$ para algún $\lambda$.}

  
\section{Desigualdad de las medias.}
  \subsection{Media geométrica y aritmética.}
    Para cualesquiera $x_1,x_2,\dots,x_n$ positivos:
    \begin{align}
      \sqrt[n]{x_1x_2 \dots x_n} \leq \frac{x_1+x_2+\dots+x_n}{n}
    \end{align}
    \equality{todos ellos son iguales, $x_1=x_2=\dots=x_n$.}
    
  \subsection{Desigualdad completa.}
    Para cualesquiera $x_1,x_2,\dots,x_n$ positivos:
    \begin{align}
      \frac{n}{\frac{1}{x_1} + \frac{1}{x_2} + \dots + \frac{1}{x_n}} \leq \sqrt[n]{x_1x_2 \dots x_n} \leq \frac{x_1+x_2+\dots+x_n}{n} \leq \sqrt{\frac{x_1^2+x_2^2+\dots+x_n^2}{n}}
    \end{align}
    \equality{todos ellos son iguales, $x_1=x_2=\dots=x_n$.}
    
  \subsection{Desigualdad generalizada.}  
    Para cualesquiera $x_1,x_2,\dots,x_n$ positivos, la función:
    \begin{align}
     f(a) = \left\{ \begin{matrix} \left(\frac{x_1^a+x_2^a+\dots+x_n^a}{n}\right)^{1/a} & si & a \neq 0 \\ \displaystyle \sqrt[n]{x_1x_2 \dots x_n} & si & a = 0\end{matrix} \right.
    \end{align}
    es creciente. \\
    \equality{será además monótona cuando $x_1=x_2=\dots=x_n$.}

    
\end{document}